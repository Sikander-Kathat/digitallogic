\documentclass{article}
\usepackage[utf8]{inputenc}
\usepackage{tikz}
\usetikzlibrary{calc}
\usepackage{float}
\usepackage{graphicx}
\usepackage{listings}
\usepackage{karnaugh-map}
\title{Assignment 9 (GATE, EC2018,18)}
\author{SIKANDER KATHAT }
\date{25 December 2020}

\begin{document}

\maketitle

\section{ Question  18 MUX diagram }
\begin{figure}[!ht]
\centering
{
\input{figure3.tex}
}
\caption{question MUX diagram}
\label{MUX diagram)}
\end{figure}

\section{Question 18}
A 4:1 multiplexer is to be used for generating the output carry of a full adder. A and B are the bits to be added while $C_i_n$ is the input carry and $C_o_u_t$ is the output carry. A and B are to be used as select bits with A being more significant select bit.

Which one of the following statement correctly describes the choice of signals to be connected to the inputs  $ I_0, I_1, I_2 and I_3 $  so that the output is $C_o_u_t$  ?

\begin{enumerate}
    \item$I_0=0,  I_1=C_i_n,  I_2=C_i_n,  I_3=1$
    \item$I_0=1,  I_1=C_i_n,  I_2=C_i_n,  I_3=1$
    \item$I_0=C_i_n,  I_1=0,  I_2=1,  I_3=C_i_n$
    \item$I_0=0,  I_1=C_i_n,  I_2=1,  I_3=C_i_n$
\end{enumerate}

\section{Solution}
\begin{table}[!ht]
\centering
{
\input{table2.tex}
}
\caption{TRUTH TABLE}
\label{table}
\end{table}


\begin{figure}[!ht]
\centering
{
\input{figure1.tex}
}
\caption{k-map for Sum}
\label{kmap Sum}
\end{figure}


    
    Boolean expression for Sum  -
    
    
    Sum=A$\overline{B}$ $\overline{C_i_n}$ + $\overline{A}$ $\overline{B}$C_i_n$ + $\overline{A}$ B $\overline{C_i_n}$ + AB$C_i_n$
    
\newpage

\begin{figure}[!ht]
\centering
{
\input{figure2.tex}
}
\caption{k-map for $C_o_u_t$}
\label{kmap $C_o_u_t$}
\end{figure}
 
 Boolean expression for $C_o_u_t$  -

$C_o_u_t$= B$C_i_n$ + AB + A$C_i_n$

\begin{table}[!ht]
\centering
{
\input{table1.tex}
}
\caption{TRUTH TABLE for $C_o_u_t$ to be output}
\label{table 1 }
\end{table}
So, by the truth table for $C_o_u_t$ to be output,
we get 


$I_0=0,  I_1=C_i_n,  I_2=C_i_n,  I_3=1$
\end{document};


